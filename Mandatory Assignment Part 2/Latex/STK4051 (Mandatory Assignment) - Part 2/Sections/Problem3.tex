\nsection{Problem 3 - (McMC – in Bayesian analysis)}
We will in this exercise consider a Bayesian
generalized linear model for identifying influential factors for presence vs absence of malaria
in blood samples taken from children in Gambia. The data set “gambia” from the geoR package
is available on the course page. A description of the data is given in an associated file. These
data are often used for spatial analysis, but we will not consider that aspect here. We will
assume all observations to be independent and investigate a probit-link between the
explanatory variables $\boldsymbol{x}$ and a binary response variable $y$. For person $i$ the probit-link between
explanatory variables and the response is defined as:
\begin{align}
    \mathbb{P}(y_i = 1) = \Phi(\boldsymbol{\beta}^T \boldsymbol{x}_i)
\end{align}
\emph{where $\Phi$ is the cumulative distribution for a standard normal variable. In the Bayesian analysis below we will use the improper prior $p(\boldsymbol{\beta}) \propto 1$, in which case the posterior $p(\boldsymbol{\beta} | \boldsymbol{y})$ is
proportional to the likelihood $L(\boldsymbol{\beta} | \boldsymbol{y})$.}
\nssection{a.)}
\emph{Define $p_i = \Phi(\boldsymbol{\beta}^T \boldsymbol{x}_i)$ and derive the likelihood function:} 
\begin{align}
    L \left(\boldsymbol{\beta}| \boldsymbol{y} \right) = \prod_{i=1}^{n} p_{i}^{y_i}(1 - p_i)^{1 - y_i} \label{eq:likelihood_exercise_3}
\end{align}
\emph{Discuss how you can implement a numerically robust evaluation of this likelihood.
How should you handle the situation where $\boldsymbol{\beta}^T \boldsymbol{x}_i$ has a large absolute value? In a Metropolis Hastings algorithm, you are asked to evaluate the ratio of two likelihoods, how is this done in a numerically stable way?}\spaze
\textbf{Solution:} \spaze

\nssection{b.)}  
\emph{Below you will be asked to implement a random walk using a random scan Metropolis
Hastings algorithm to investigate the posterior distribution. From the general
expression for the Metropolis Hastings (M-H) ratio, deduce the M-H ratio for the
random walk. Which criteria need to be met in order for the Markov chain to converge
to the stationary distribution? Which of these criteria does the M-H ratio help you to
fulfill?}  \spaze
\textbf{Solution:} \spaze

\nssection{c.)}
\emph{For the Gambia data: Implement a random walk algorithm with a random scan to
sample from the posterior distribution $p(\boldsymbol{\beta} | \boldsymbol{y})$, and apply it. Use a model containing the explanatory variables: age, netuse, treated, green, and phc, in addition to the constant term. (Hint: 1-Remember to standardize the design matrix, 2- put some effort
into a robust evaluation of the likelihood ratio, see a)}  \spaze
\textbf{Solution:} \spaze

\nssection{d.)}
\emph{Display plots which illustrate the convergence properties of the method. Comment on
the convergence properties of your algorithm. If there are any obvious problems,
suggest modifications and revisit c to improve convergence.} \spaze 
\textbf{Solution:} \spaze

\nssection{e.)}
\emph{A common way to sample the distribution above is to introduce a latent variable, $z_i$, which is
defined such that $\bl y_i = 1 \br \iff \bl z_i > 0\br$, and}
\begin{align}
    z_i \sim \mathcal{N} \left( \boldsymbol{\beta}^T \boldsymbol{x}_i, 1^2 \right)
\end{align}
\emph{Argue that the expanded posterior probability distribution}
\begin{align*}
p\left(\boldsymbol{\beta}, \boldsymbol{z} | \boldsymbol{y} \right) \propto  \prod_{i=1}^{n} \left[I(z_i > 0) \cdot I(y_i = 1) + I(z_i \leq 0) I(y_i=0) \right] \phi \left( z_i - \boldsymbol{\beta}^T \boldsymbol{x}_i \right)
\end{align*}
\emph{will have the prescribed marginal posterior, $p(\boldsymbol{\beta} | \boldsymbol{y})$ from expression (\ref{eq:likelihood_exercise_3}). (Hint: consider one data point first and integrate out the latent variable.)} \spaze
\textbf{Solution:} \spaze

\nssection{f.)}
\emph{You shall now derive the conditional distributions needed for Gibbs sampling. Show
that:}
\begin{align*}
    p(z_i | \boldsymbol{z}_{-i}, \boldsymbol{\beta}, \boldsymbol{y}) \propto \begin{cases}
      I(z_i \leq 0) \phi(z_i - \boldsymbol{\beta} \boldsymbol{x}_i),   & \text{ if } y_i = 0 \\[5pt]
      I(z_i > 0) \phi(z_i - \boldsymbol{\beta} \boldsymbol{x}_i),   & \text{ if } y_i = 1
    \end{cases}
\end{align*}
\emph{and} 
\begin{align*}
    p(\boldsymbol{\beta} | \boldsymbol{z}, \boldsymbol{y}) \propto \text{exp} \left(-0.5 \sum_{i=1}^{n} (z_i - \boldsymbol{\beta}^T \boldsymbol{x}_i)^2 \right)
\end{align*}
\emph{Use results from multi linear regression to deduce that the distribution $p(\boldsymbol{\beta} | \boldsymbol{z}, \boldsymbol{y})$, is multi-normal with parameters:}
\begin{align*}
    \mathbb{E}(\boldsymbol{\beta} | \boldsymbol{z}, \boldsymbol{y}) = \left(\boldsymbol{X}^T \boldsymbol{X} \right)^{-1}\boldsymbol{X}^T \boldsymbol{z}
\end{align*}
\emph{and}
\begin{align*}
    \text{Cov}(\boldsymbol{\beta} | \boldsymbol{z}, \boldsymbol{y}) = (\boldsymbol{X}^T \boldsymbol{X})^{-1}
\end{align*}
\emph{State the result from multivariate linear regression you are using and show how you use
the connection to derive the results.} \spaze
\textbf{Solution:} \spaze

\nssection{g.)}
\emph{ Implement the Gibbs sampler, and use this to solve the inference of Gambia data. Use
a block update to sample $p(\boldsymbol{\beta} | \boldsymbol{z}, \boldsymbol{y})$. Make illustrations of the convergence of this chain as well (as done for the random walk approach in in d). (Hint: to sample from the multi-normal distribution use code from a library, e.g. \texttt{rmvnorm} from \texttt{mvtnorm}} \spaze 
\textbf{Solution:} \spaze

\nssection{h.)}
\emph{Compare the two approaches you have implemented for evaluating the posterior.
Compare implementation, runtime effects, and results e.g. burn in, convergence, mixing
etc.} \spaze 
\textbf{Solution:} \spaze